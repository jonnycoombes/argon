\documentclass{book}
\usepackage{titling}
\usepackage[hidelinks]{hyperref}
\usepackage{amsmath}
\usepackage{amssymb}

\newcommand{\subtitle}[1]{%
    \posttitle{%
    \par\end{center}
    \begin{center}\large#1\end{center}
    \vskip0.5em}%
}


\title{Glencore}
\subtitle{Argon Development Notes}
\author{Jonny Coombes}
\date{November 2020}

\begin{document}


\maketitle
\pagenumbering{gobble}
\newpage
\tableofcontents
\newpage
\pagenumbering{arabic}

\chapter{General Notes}
\section{Notes on Versioning}
The general versioning scheme for Argon builds will comprise of three main components:
\begin{enumerate}
    \item The \textit{major} version number.
    \item The \textit{minor} version number.
    \item Either:
    \begin{enumerate}
        \item A specific build identifier (monotonically increasing).
        \item A specific patch/correlated fix identifier.
    \end{enumerate}
\end{enumerate}

\section{Source Control Notes}
The Git versioning scheme for Argon is straightforward:
\begin{enumerate}
    \item The main development branch is \textit{mainline}.
    \item Each intermediate release will have a specific branch, named in accordance with the release.  Currently planned intermediate releases are as follows:
    \begin{enumerate}
        \item 0.1.0 - build after the initial development sprint.
        \item 0.2.0 - build after the second development sprint.
        \item 0.3.0 - build after the third development sprint.
    \end{enumerate}
    \item Individual feature implementations will be carried out on a dedicated branch, prefixed with the corresponding \textbf{Jira} ticket.  For example, ticket number \textbf{JA-15} would have a branch named \textit{feature\\JA-15-Summary}, where the summary is automatically generated as part of the development toolchain.
    \item Within the local development environment, changes are mastered and then pushed to multiple remotes.  (There may be multiple remotes based on the number of environments stood up).
\end{enumerate}

\section{Library Dependencies}
The key libraries used throughout the build of the Argon project as given in table \ref*{table:1} below:\\
\begin{table}[h!]
    \centering
    \begin{tabular}{||l | c | l ||}
        \hline
        \textbf{Library} & \textbf{Version} & \textbf{Description} \\
        \hline
        .NET Core & 5.0 & Core .NET platform runtime\\
        ASP.NET & 5.0 & ASP.NET Core library \\
        EF Core & 5.0 & EF framework (plus RDBMS specifics)\\
        Serilog & 2.10.0 & Logging library\\
        Serilog.Sinks.Console & 3.1.1 & Console sink for Serilog \\
        Polly & X.X & Policy library\\
        \hline 
    \end{tabular}
    \caption{Key Argon Dependencies}
    \label{table:1}
\end{table}

\end{document}
