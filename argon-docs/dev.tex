\documentclass{book}
\usepackage{titling}
\usepackage[hidelinks]{hyperref}
\usepackage{amsmath}
\usepackage{amssymb}
\usepackage{fancyhdr}
\usepackage[a4paper]{geometry}
\usepackage{layout}
\usepackage{float}

\title{Glencore Development Notes}
\author{Jonny Coombes}
\date{November 2020}

\begin{document}
\maketitle
\pagenumbering{gobble}
\newpage
\tableofcontents
\let\cleardoublepage=\clearpage
\newpage
\pagenumbering{arabic}

\chapter{General Notes}
\hoffset=0in
% The default is to have differing margins dependent on even or odd pages
\evensidemargin=0in
\section{Versioning}
The general versioning scheme for Argon builds will comprise of three main components:
\begin{enumerate}
    \item The \textit{major} version number.
    \item The \textit{minor} version number.
    \item Either:
    \begin{enumerate}
        \item A specific build identifier (monotonically increasing).
        \item A specific patch/correlated fix identifier.
    \end{enumerate}
\end{enumerate}

\section{Source Control}
The Git versioning scheme for Argon is straightforward:
\begin{enumerate}
    \item The main development branch is \textit{mainline}.
    \item Each intermediate release will have a specific branch, named in accordance with the release.  Currently planned intermediate releases are as follows:
    \begin{enumerate}
        \item 0.1.0 - build after the initial development sprint.
        \item 0.2.0 - build after the second development sprint.
        \item 0.3.0 - build after the third development sprint.
    \end{enumerate}
    \item Individual feature implementations will be carried out on a dedicated branch, prefixed with the corresponding \textbf{Jira} ticket.  For example, ticket number \textbf{JA-15} would have a branch named \textit{feature\\JA-15-Summary}, where the summary is automatically generated as part of the development toolchain.
    \item Within the local development environment, changes are mastered and then pushed to multiple remotes.  (There may be multiple remotes based on the number of environments stood up).
\end{enumerate}

\section{Build Dependencies}
The key libraries used throughout the build of the Argon project as given in table \ref{table_dependencies} below:
\begin{table}[h!]
    \centering
    \begin{tabular}{||l | c | l ||}
        \hline
        \textbf{Library} & \textbf{Version} & \textbf{Description} \\
        \hline
        .NET Core & 5.0 & Core .NET platform runtime\\
        ASP.NET & 5.0 & ASP.NET Core library \\
        EF Core & 5.0 & EF framework (plus RDBMS specifics)\\
        Serilog & 2.10.0 & Logging library\\
        Serilog.Sinks.Console & 3.1.1 & Console sink for Serilog \\
        Polly & X.X & Policy library\\
        \hline 
    \end{tabular}
    \caption{Key Argon Dependencies}
    \label{table_dependencies}
\end{table}

\section{Database Notes}
General DB notes:
\begin{enumerate}
    \item \textbf{Development SQL Server Version (Ryleh)}: Microsoft SQL Server 2019 (RTM) - 15.0.2000.5 (X64) (Sep 24 2019 13:48:23)
    \item \textbf{Development Argon Login Name} : \textit{argon}
    \item \textbf{Development Argon Db Name}: \textit{argon}
    \item \textbf{Core Schema Name}: \textit{core} 
\end{enumerate}
The nominated user used to connect through via the DB context, must be a member of the following SQL Server roles:
\begin{enumerate}
    \item dbcreator
\end{enumerate}
This allows for just a user login to be allocated on the target SQL instance, and then Argon can take care of creating the necessary database/schema objects.  (\textit{This adheres to the principal that configuration should be a close to zero as possible for new deployments}).
\subsection{Core Tables}
\begin{table}[H]
    \centering
    \begin{tabular}{|| p{0.1\linewidth} | p{0.2\linewidth} | p{0.7\linewidth} ||}
        \hline
        \textbf{Schema} & \textbf{Table} & \textbf{Description}\\
        \hline
        core & collection & The main collections table, one row per collection.\\
        \hline
        core & constraint & The collection constraints table. $1 \rightarrow *$ relationship between entries in the \textit{collection} table and entries within this table. \\
        \hline
        core & item & The collection items table.  $1 \rightarrow +$ relationship between the \textit{collection} table and entries within this table.\\
        \hline
        core & version & The collection item versions table. $1 \leftarrow +$ relationship between entries in this table and the items table.  (Each item will have a minimum of one version).\\
        \hline
        core & propertyGroup & The main property group table there is a $1 \rightarrow *$ relationship between this table and the property table.\\
        \hline
        core & property & The property table. Each item (and potentially) collection will have multiple associated rows within this table, through a given property group link. \\
        \hline 
    \end{tabular}
    \caption{Core Tables}
    \label{table_core_tables}
\end{table}

\subsection{EF Core Specifics}
Some general notes on the EF Core implementation within Argon:
\begin{enumerate}
    \item All useful model entities will utilise a \textit{Timestamp} as a concurrency token. 
\end{enumerate}

\section{AWS Deployments}
Current AWS deployment details in table \ref{table:2} below:
\begin{table}[h!]
    \centering
    \begin{tabular}{|| l | l | l | l ||}
        \hline 
        \textbf{Component} & \textbf{Type} & \textbf{Name} & \textbf{Details} \\
        \hline
        VPC & Virtual Private Cloud & Argon-Dev-VPC & Partitioned VPC \\
        Subnet & VPC Subnet & Argon-Dev-SN-1 & 10.0.0.0/24 CIDR \\
        Sec. Group & Security Group & Argon-Dev-SG-1 & SSH, HTTPS, SQL (Restricted) \\
        Internet Gateway & (E/I) Gateway & Argon-Dev-IG & Ingress and outgress \\
        EIP & Elastic IP & No public DNS yet & 34.249.105.124 \\
        Host & t3a.small & argon-dev-1 & Ubuntu 18.04 LTS \\
        \hline
    \end{tabular}
    \caption{AWS Deployment Artifacts}
    \label{table:2}
\end{table}   

\section{Code Notes}
\subsection{Namespaces}
The general layout of the Argon core code adheres to the following conventions:
\begin{enumerate}
    \item The top level namespace for the core is \textbf{JCS.Argon}
    \item Key code artifacts are organised so that:
    \begin{enumerate}
        \item Controllers are placed in the \textbf{JCS.Argon.Controllers} namespace.
        \item Services are placed in the \textbf{JCS.Argon.Services} namespace.
        \item Model elements have a top-level namespace of \textbf{JCS.Argon.Model}.
    \end{enumerate}
\end{enumerate}
\subsection{OpenAPI Endpoints}
The OpenAPI specification that the API conforms to is published at the following location (relative to the deployment root):
\begin{center}
    \url{/swagger/v1/swagger.json}
\end{center}
The version path component is expected to remain constant during the initial release, and will only change with \textit{significant} breaking change releases in the future.
All initial API endpoints will be prefixed as follows:
\begin{center}
    \url{http[s]://[host]:[port]/api/v1}
\end{center}
\subsubsection{Cloud Hosted Endpoints}
During development instances of the current build/deployed artifacts are available at the following locations:
\begin{center}
    \url{http[s]://argon-dev-1.jcs.software.co.uk/api/v1}
    \url{http[s]://argon-dev-1.jcs-software.co.uk/swagger/v1/swagger.json}
\end{center}
\chapter{Release Notes}
\section{Build 0.1.15}
\textbf{Deploy Date 23\textsuperscript{rd} November, 2020}
\paragraph{}
This release represents the first stable build of the Argon core with a limited feature set.  Within this release:
\begin{enumerate}
    \item The core database schema is largely complete, however still in a state of flux, given that complete implementation of the constraint and property group abstractions is not yet complete.
    \item The core interfaces around the \textit{VSP} subsystem are defined and 95\% implemented for the \textit{Native File} VSP provider.  (Still subject to change throughout sprints 2 and 3).
    \item The general API surface area and associated Swagger definitions is approximately 95\% complete.
    \item Asynchronous background processing for item checksums is not yet implemented.
    \item Not authentication is present within this build. 
\end{enumerate}
\section{Build 0.2.1}
\textbf{Deploy Date 24\textsuperscript{th} November, 2020}
\section{Build 0.2.2}
\textbf{Deploy Date 25\textsuperscript{th} November, 2020}
\end{document}
